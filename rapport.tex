\documentclass{article}
\usepackage{graphicx} % Required for inserting images
\usepackage{float} % For the H specifier
\author{Bakkari Ikrame\and{Monnier Killian}\and{André Émilien}}
\title{Quantum Machine Learning Based Intrusion Detection System}
\begin{document}
\maketitle
\section{Abstract}
An abstract is a brief summary of a research article, thesis, review, 
conference proceeding, or any in-depth analysis of a particular subject and is 
often used to help the reader quickly ascertain the paper's purpose.equation
\section{Introduction}
As cyberattacks become more sophisticated, current IDS doesn't match
\section{State of the Art}
Currently, a qubit lifetime vary from dozens to hundreds of milliseconds. [1]
Thus, qubit error correction is mandatory. The five-qubit error correcting 
code is the smallest quantum error correcting code that can protect a logical 
qubit. [2]\par
IBM Condor processor has 1,121 logical qubits. The cooler the processor, 
the more the qubits are stable. Thus, the Condor needs to be cooled down to a 
temperature of around 0.015 K. Today we can freely use IBM 127-qubit Eagle processor 
for 10 minutes per month. Heron is their fastet quantum processor with 133 
qubits and they are actively working on coupling samples of it. They call this 
technology the IBM Q System Two. IBM claims this quantum computer will advance 
science. [3]
\section{Our QML based IDS}
If we keep all the 41 features of the NSL-KDD dataset, our Quantum model will
have 164 weights. If we sum these weights with the 41 features we might need
a machine with 205 qubits. And, no such machine can be rented from IBM.\par
So, we need to reduce our number of features. Our approach is to train a
classical model with the NSL-KDD and compute the Shapley values of each
features for predicting the testing dataset.\par
The Shapley value comes from the game theory and measures the contribution of 
a player (a feature) in a coalition (all the features) to the game's result 
(the prediction). It is used and reknown in Explainable Artificial 
Intelligence.\par
If we define the number of qubits our model needs to be at least five times 
the number of features (it might needs other qubits we don't know yet), and, 
we consider we can rent a 127-qubit processor from IBM we can afford 20 
features (we keep 27 qubits if we need them later).\par
We defined the number of features to keep and how to select them given a model
and a test dataset. Now we will define how we selected our model.\par
To simplify things, if a stream is normal its label is zero, if it is not, its 
label is one. Thus, our model is a binary classifier.\par
The recall of a binary classifier is defined as
$$\frac{true\_positives}{true\_positives+false\_negatives}$$
If our recall is one, then, we detect every intrusion but we might also detect
everything as an intrusion and the human operator monitoring the IDS might miss
a real intrusion. We need not to overwelm the monitor with false negatives. 
So, we evaluated the precision which measures the true intrusions 
rate among streams detected as intrusions. The F1 score combines both the 
recall and the precision. Thus, we select the model with the best F1 score.\newline
\begin{table}[H]
\caption{The classical classifiers evaluations}
\label{f1score-classical-classifier}
\center
\begin{tabular}{|l|l|}
\hline
        Classifier & F1 Score \tabularnewline
        \hline
        Logistic Regression &  65.86\% \tabularnewline
        Random Forest & 74.95\% \tabularnewline
        Decision Tree & 82.30\% \tabularnewline
        KNN & 75.67\% \tabularnewline
        Naive Bayes & 7.09\% \tabularnewline
\hline
\end{tabular}
\end{table}
%TODO The 5 first features with the best Shapley values when classifying KDDTest+.txt's streams with our fitted decision tree classifier
Our data is classic and our algorithm is quantum. So, we use a second-order 
Pauli-Z evolution circuit alias ZZ feature map to encode our data.
\section{Results}
\section{Conclusion}
\section{References}
[1] : https://www.nature.com/articles/s41534-021-00510-2\newline
[2] : Gottesman, Daniel (2009). "An Introduction to Quantum Error Correction 
and Fault-Tolerant Quantum Computation".\newline
[3] : https://www.youtube.com/watch?v=Qndz54SGCAs
[4] : https://newsroom.ibm.com/IBM-research?item=32425
[5] : https://techhq.com/2020/12/ibm-super-fridge-aims-to-solve-quantum-computer-cooling-problem/
\end{document}
